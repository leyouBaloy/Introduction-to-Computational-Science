\documentclass{article}
\usepackage[UTF8]{ctex}
\usepackage{geometry}
\usepackage{natbib}
\geometry{left=3.18cm,right=3.18cm,top=2.54cm,bottom=2.54cm}
\usepackage{graphicx}
\pagestyle{plain}	
\usepackage{setspace}
\usepackage{caption2}
\usepackage{float}
\usepackage{datetime} %日期
\renewcommand{\today}{\number\year 年 \number\month 月 \number\day 日}
\renewcommand{\captionlabelfont}{\small}
\renewcommand{\captionfont}{\small}
\begin{document}

\begin{figure}
    \centering
    \includegraphics[width=8cm]{upc.png}

    \label{figupc}
\end{figure}

	\begin{center}
		\quad \\
		\quad \\
		\heiti \fontsize{45}{17} \quad \quad \quad 
		\vskip 1.5cm
		\heiti \zihao{2} 《计算科学导论》课程总结报告
	\end{center}
	\vskip 2.0cm
		
	\begin{quotation}
% 	\begin{center}
		\doublespacing
		
        \zihao{4}\par\setlength\parindent{7em}
		\quad 

		学生姓名:\underline{\qquad  孙百乐 \qquad \qquad}

		学\hspace{0.61cm} 号:\underline{\qquad 2007010218\qquad}
		
		专业班级:\underline{\qquad 人工智能2001 \qquad  }
		
        学\hspace{0.61cm} 院:\underline{计算机科学与技术学院}
% 	\end{center}
		\vskip 2cm
		\centering
		\begin{table}[h]
            \centering 
            \zihao{4}
            \begin{tabular}{|c|c|c|c|c|c|c|}
            % 这里的rl 与表格对应可以看到,姓名是r,右对齐的;学号是l,左对齐的;若想居中,使用c关键字。
                \hline
                课程认识 & 问题思 考 & 格式规范  & IT工具  & Latex附加  & 总分 & 评阅教师 \\
                30\% & 30\% & 20\% & 20\% & 10\% &  &  \\
                \hline
                 & & & & & &\\
                & & & & & &\\
                \hline
            \end{tabular}
        \end{table}
		\vskip 2cm
		\today
	\end{quotation}

\thispagestyle{empty}
\newpage
\setcounter{page}{1}
% 在这之前是封面,在这之后是正文
\section{引言}
大一上学期,由孙运雷老师为我们讲授了《计算科学导论》这门课。除了讲课外,孙老师还布置了分组演讲任务,课程结束后,我收获不少,现总结如下。

\section{对计算科学导论这门课程的认识、体会}

《计算科学导论》从科学哲学的角度出发,系统地介绍了计算科学的定义,特点,范畴,形态,历史渊源,发展变化,知识组织结构和分类体系,学科专业培养模式和课程体系等内容,有助于学生比较全面的了解计算科学。\par 
研究计算科学,应该是非常难的,但是意义重大。首先可以用于计算机,哪些问题是计算机可以解决的?哪些问题计算机不可能解决?如果计算机可以解决,那么它该如何解决?怎么样让计算机计算问题更快效率更高?其次,计算科学能帮助我们认识自然,如果物理世界是可以计算的,那么我们是否可以通过计算来模拟世界?人类的科学几乎全是建立在计算之上,经验可能会出错,计算会出错吗?这些问题,很值得人类研究。\par 
计算科学导论这门课放在大一,说白了就是为了让刚进入大学的新生对自己大学四年要学的内容有整体认识。但是这本书讲的内容还是比较枯燥难懂的,有很多概念是我们从未见过的,比如什么“计算机语义学”,“图灵机”等等,再加上这本书干货很多,书里概括性的一句话都可能拿出来再写一本书,所以我们理解起来还是很困难的。不过还好,有孙老师为我们讲解这门课,他知识广博,思维跳跃,讲课深入浅出,使我们在对课本内容有了解之外,还能学到其它知识。\par 
上完这个课,当别人问我学计算机是不是就是学修电脑的,我可以很肯定的告诉她我学的不只是修电脑。\par 


\subsection{同学们的分组演讲}
计导课还有一个很重要的部分就是分组演讲,孙老师弄了1500多个课题给同学们选择,很多同学讲的都很有意思。\par 
我们班第一组演讲课题是“人流检测”,最开始的人流检测用的是闸机,是纯机械装置,现在依然很常见,在石大图书馆,青岛海洋馆,北京天安门广场我都见到过。现在有了新了选择,比如商场可以用wifi,人脸识别等技术实现无感知地检测人流量。生活中习以为常的东西,我不以为意。但其实每一项技术都有它的发展历史和未来趋势。这启示了我,不要总是想在大的方面取得成就,比如研究深度学习,这太难了。不如把眼光放低,把某项小的技术做到NO.1,也很不错。\par
还有同学演讲题目是指纹识别,虹膜识别等技术。这两种技术其实很早就诞生了,但是指纹识别近几年发展迅速,而虹膜识别确一直不上台面,这是为什么呢?因为指纹识别才是真正的需求。苹果第一个把指纹识别做到了手机上,消费者惊呆了,居然这么好用!这也是苹果的伟大之处,它能做到“生产决定需求”。从此以后,各个厂商纷纷效仿,在手机上加入指纹,指纹技术越来越先进,价格却越来越便宜。现在还有超声波指纹,可以实现屏下指纹识别。虹膜识别本来就有难以攻克的问题,再加上指纹识别已经抢占了全壁江山,现在再去做虹膜识别就是自寻死路。\par 
孙老师说,判断力十分重要,我们要判断什么才是真需求,什么是假需求。判断这个其实不太容易,当年西方人第一次到非洲时,看见非洲人不穿鞋,以为发现了巨大商机。但其实非洲人根本不需要鞋,所以鞋对非洲人来说就是假需求。英国人工业革命后开辟中国市场,往中国卖纺织品,中国是纺织大国,对这玩意也没什么需求,所以英国人卖不出去只能卖鸦片。现在时代变化的这么快,浪潮是一波接一波,有的浪潮很高很大,有的浪潮很小很矮,就是如果我们具有了判断力,就能够勇立潮头,经久不衰。雷军有一个“风口上的猪理论”,很有意思,就是说:如果你站在风口上,即使你是一头猪也能起飞。所以上完课后我知道了判断力是我需要培养的。\par 
还有同学讲到人工智能下围棋,孙老师补充说围棋已经被人工智能玩到顶了,目前来看,人类是不可能在围棋方面打过人工智能了。围棋是一种玩家有全知视角的游戏,现在谷歌的团队在搞能打非全知视角游戏的人工智能,比如说打星际争霸啥的,这样的游戏有战争迷雾,有一定的未知性,有的时候赢了不是靠技术而是靠运气。我觉得这玩意还是挺可怕的,现在电竞那么火,万一这个也被人工智能攻克了,那肯定会触犯到一大波人的利益。所以我觉得这个不是很好的发展方向。\par 
我们现在还处于弱人工智能时代,现在人工智能的局势是:在一些方面能很好代替人,但在绝大多数领域,人工智能在人类眼里还是人工智障,不可能取代人。我觉得现在这个时代还挺好,万一哪天人工智能真发展到有意识的那种,像电影里演的那样早饭把人类干掉了那就完蛋了。\par 

\subsection{对本研班的看法}
我觉得进本研班的一大好处就是避免了“卷”的风气。我观察到,在其它班级里会有同学还保持高中的思维,对成绩有一种执念。有的人在寝室里展现出自己不学习的样子,然后在自习室里“偷学”。一口一个“大佬”称呼别人,实际上心里非常害怕别人超过自己。这样的人,继承了高中的传统,仍然觉得成绩最重要。同时又目光短浅,不立志考全校第一,只求比寝室里的同学考得好就行。这也是正常现象,毕竟其它班级有考研,保研的压力。这样的好处是:保持了积极的学习态度,至少比那些刚进入大学就堕落的人强。坏处就是:进入大学之后并没有明确的目标,不知道自己真正想要干什么,就只能在成绩上较劲。\par 
本研班没有争保研名额的压力,所以有一部分同学貌似在混日子,没看出来他想干什么。好处是大家都相处的很融洽,没有任何“窝里斗”的迹象。大家都在寻找自己感兴趣的事做,有的人喜欢智能车,有的人喜欢打acm,有的人喜欢打篮球,甚至他已经在学校里当裁判了。如果大家都能有自己的想法,尽早投身到自己喜欢的领域,就更有可能做出更大的成就来。我挺喜欢本研班的,我理想中的班级氛围应该是:上课的时候大家一起认真学习,打好数理基础。下课时分成若干个小组去做一些实践项目,在实践中学习一些课堂学不到的东西。我呆在这个班级里,我希望我身边能多一些积极上进的人,我们一起搞“大事情”。\par 
孟子说:“穷则独善其身,达则兼济天下。”我现在有的是时间和精力,我想以我的方式多参与到班级事务当中,协助班长一起营造良好的班级氛围。我很喜欢何为班长,他在当班长方面,可以说算是一个“专家”,是有天分的,值得我信赖,我也能从他身上学到东西。不过我这个年龄的人,思想还很幼稚,做事情是“摸着石头过河”,走一步算一步。过早选择未来的道路,我很有可能会选错,不过没关系,孔子说:“知其不可而为之”,我觉得孔圣人之所以被称为是“圣人”,就是因为他有这种情怀。我也要有这样的情怀,我觉得本研班也要有这样的情怀。\par 
\section{进一步的思考}
结合学习的计算科学知识,对分组演讲涉及的问题作进一步的思考。\par

这里是简单列表的样例:(如果需要标号自定义或者自动标记数字序号,请自行搜索语法)
\begin{itemize}
    \item 简单的列表结构 
    \item 如这里所示
    \item 此处仅为样例
    \item 按需修改和使用
\end{itemize}

\section{总结}
在这里,写自己对于整个课程和或本次报告的总结。\par


\section{附录}
\begin{itemize}
    \item 申请Github账户,给出个人网址和个人网站截图
    \item 注册观察者、学习强国、哔哩哔哩APP,给出对应的截图
    \item 注册CSDN、博客园账户,给出个人网址和个人网站截图
    \item 注册小木虫账户,给出个人网址和个人网站截图
\end{itemize}


\hspace*{\fill} \\

{\bf 注意,参考文献至少五篇,其中至少两篇为英文文献,参考文献必须在正文中有引用。}
\bibliographystyle{plain}
\bibliography{references}


\end{document}
